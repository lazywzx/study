\documentclass[12pt,a4paper]{article}
\usepackage{graphicx}
\usepackage{ctex}
\usepackage{latexsym}
\usepackage{indentfirst}
\usepackage{subfigure}
\usepackage{amssymb}
\usepackage{CJK}
\usepackage{cite}
\usepackage{amsmath}%数学
\usepackage{titlesec}%设置章节标题与正文间距为2行
\usepackage{enumerate}%项目编号
\usepackage[]{caption2}%去掉图片编号后的":"
\usepackage{booktabs}%表格用
\usepackage{titlesec}%修改标题格式宏包
\usepackage{multirow}%跨行表格
\usepackage{abstract}%摘要
\usepackage{setspace}   %行间距的宏包
\usepackage{makecell}%表格竖线连续
\usepackage{float}%可以用于禁止浮动体浮动
\usepackage[colorlinks,linkcolor=black,anchorcolor=blue,citecolor=black]{hyperref}%目录超链接
\usepackage{listings}%可以插入代码
\usepackage{xcolor}%语法高亮支持
\usepackage[left=3.0cm, right=2.6cm, top=2.54cm, bottom=2.54cm]{geometry}%设置页面格式
%代码格式
\definecolor{dkgreen}{rgb}{0,0.6,0}
\definecolor{gray}{rgb}{0.5,0.5,0.5}
\definecolor{mauve}{rgb}{0.58,0,0.82}
\usepackage{fontspec}
\setmonofont{Consolas}
\lstset{ %
	numbers=left, 
	basicstyle=\tiny\ttfamily, 
	numberstyle=\tiny, 
	tabsize=4,
	numbersep=5pt, 
	keywordstyle= \color{blue!70}, %关键词为蓝色
	commentstyle=\color{gray}, %注释为灰色
	frame=shadowbox, % 框架阴影效果
	rulesepcolor= \color{ red!20!green!20!blue!20} ,
	escapeinside={\%*}{*)},
	xleftmargin=2em, % 边界选项
	xrightmargin=2em, % 边界选项
	aboveskip=1em, % 边界选项
	framexleftmargin=2em, % 边界选项
	breaklines,%过长代码自动换行
}
\titlespacing{\section}{0pt}{0pt}{2em}
\renewcommand{\figurename}{图}%将figure改为图
\renewcommand{\captionlabeldelim}{}
\renewcommand {\thefigure} {\thesection{}.\arabic{figure}}%图片索引该为按照章节
\titleformat{\section}{\centering\zihao{3}\bfseries}{\arabic{section}.}{0.5em}{}%修改section标题格式
\def\I{\vrule width1.2pt}
%!\I 就可以代替| 来画表格了
%可固定下划线长度
\makeatletter
\newcommand\dlmu[2][4cm]{\hskip1pt\underline{\hb@xt@ #1{\hss#2\hss}}\hskip3pt}
%使\section中的内容左对齐
\renewcommand{\section}{\@startsection{section}{1}{0mm}{-\baselineskip}{0.5\baselineskip}{\bf\leftline}}
\makeatother 

\begin{document}

%封面部分
\begin{titlepage}
	\centering
	\vspace{2\baselineskip}
	{LaTex笔记}
	\vspace{2cm}

	\centering
	{	
		E-mail: \url{wangzixu98@icloud.com}
		\vspace{0.5cm}

		HomePage: \url{http://lazywzx.info}
		\vspace{0.5cm}
		
		GitHub: \url{https://github.com/lazywzx}
	}

\end{titlepage}

% %目录部分
% \renewcommand{\contentsname}{\centerline{\zihao{-2}\textbf{目录}}}
% \tableofcontents

% \newpage
% %正文部分
% \setlength{\baselineskip}{23pt}

\section{表达式与方程组}

离散求和
\begin{align}
    H(x) = \sum_{i=0, j=0}^{N-1, M-1}(x_{i, j} = x), x = 0, 1, \cdots, 255
\end{align}

离散求和(方程组)
\begin{align}
    u &= \sum_{i=0}^{k}\sum_{j=0}^{i} a_{ij} \cdot x^{i}y^{i-j} \\
    v &= \sum_{i=0}^{k}\sum_{j=0}^{i} b_{ij} \cdot x^{i}y^{i-j}
\end{align}

均方根误差RMSE
\begin{align}
    RMSE_{x} = \sqrt{\frac{\sum\limits_{i=0}^{N-1}  (x_{i}^{'} - u_{i})^{2}}{N}} \\
    RMSE_{y} = \sqrt{\frac{\sum\limits_{i=0}^{N-1}  (y_{i}^{'} - v_{i})^{2}}{N}}
\end{align}

\section{向量与矩阵}

矩阵运算
\begin{equation}
	\begin{bmatrix}
		u \\
		v
	\end{bmatrix}
	= 
	\begin{bmatrix}
		a_{0} & a_{1} & a_{2} \\
		b_{0} & b_{1} & b_{2}
	\end{bmatrix}
	\cdot
	\begin{bmatrix}
		1 & x & y
	\end{bmatrix}
	^{T}
\end{equation}

\end{document}
